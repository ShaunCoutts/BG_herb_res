% Template for PLoS
% Version 3.3 June 2016
%
% % % % % % % % % % % % % % % % % % % % % %
%
% -- IMPORTANT NOTE
%
% This template contains comments intended 
% to minimize problems and delays during our production 
% process. Please follow the template instructions
% whenever possible.
%
% % % % % % % % % % % % % % % % % % % % % % % 
%
% Once your paper is accepted for publication, 
% PLEASE REMOVE ALL TRACKED CHANGES in this file 
% and leave only the final text of your manuscript. 
% PLOS recommends the use of latexdiff to track changes during review, as this will help to maintain a clean tex file.
% Visit https://www.ctan.org/pkg/latexdiff?lang=en for info or contact us at latex@plos.org.
%
%
% There are no restrictions on package use within the LaTeX files except that 
% no packages listed in the template may be deleted.
%
% Please do not include colors or graphics in the text.
%
% The manuscript LaTeX source should be contained within a single file (do not use \input, \externaldocument, or similar commands).
%
% % % % % % % % % % % % % % % % % % % % % % %
%
% -- FIGURES AND TABLES
%
% Please include tables/figure captions directly after the paragraph where they are first cited in the text.
%
% DO NOT INCLUDE GRAPHICS IN YOUR MANUSCRIPT
% - Figures should be uploaded separately from your manuscript file. 
% - Figures generated using LaTeX should be extracted and removed from the PDF before submission. 
% - Figures containing multiple panels/subfigures must be combined into one image file before submission.
% For figure citations, please use "Fig" instead of "Figure".
% See http://journals.plos.org/plosone/s/figures for PLOS figure guidelines.
%
% Tables should be cell-based and may not contain:
% - spacing/line breaks within cells to alter layout or alignment
% - do not nest tabular environments (no tabular environments within tabular environments)
% - no graphics or colored text (cell background color/shading OK)
% See http://journals.plos.org/plosone/s/tables for table guidelines.
%
% For tables that exceed the width of the text column, use the adjustwidth environment as illustrated in the example table in text below.
%
% % % % % % % % % % % % % % % % % % % % % % % %
%
% -- EQUATIONS, MATH SYMBOLS, SUBSCRIPTS, AND SUPERSCRIPTS
%
% IMPORTANT
% Below are a few tips to help format your equations and other special characters according to our specifications. For more tips to help reduce the possibility of formatting errors during conversion, please see our LaTeX guidelines at http://journals.plos.org/plosone/s/latex
%
% For inline equations, please be sure to include all portions of an equation in the math environment.  For example, x$^2$ is incorrect; this should be formatted as $x^2$ (or $\mathrm{x}^2$ if the romanized font is desired).
%
% Do not include text that is not math in the math environment. For example, CO2 should be written as CO\textsubscript{2} instead of CO$_2$.
%
% Please add line breaks to long display equations when possible in order to fit size of the column. 
%
% For inline equations, please do not include punctuation (commas, etc) within the math environment unless this is part of the equation.
%
% When adding superscript or subscripts outside of brackets/braces, please group using {}.  For example, change "[U(D,E,\gamma)]^2" to "{[U(D,E,\gamma)]}^2". 
%
% Do not use \cal for caligraphic font.  Instead, use \mathcal{}
%
% % % % % % % % % % % % % % % % % % % % % % % % 
%
% Please contact latex@plos.org with any questions.
%
% % % % % % % % % % % % % % % % % % % % % % % %

\documentclass[10pt,letterpaper]{article}
\usepackage[top=0.85in,left=2.75in,footskip=0.75in]{geometry}

% amsmath and amssymb packages, useful for mathematical formulas and symbols
\usepackage{amsmath,amssymb}

% Use adjustwidth environment to exceed column width (see example table in text)
\usepackage{changepage}

% Use Unicode characters when possible
\usepackage[utf8x]{inputenc}

% textcomp package and marvosym package for additional characters
\usepackage{textcomp,marvosym}

% cite package, to clean up citations in the main text. Do not remove.
\usepackage{cite}

% makes the \textsubscript work on my older version of latex
\usepackage{fixltx2e}

% Use nameref to cite supporting information files (see Supporting Information section for more info)
\usepackage{nameref,hyperref}

% line numbers
\usepackage[right]{lineno}

% ligatures disabled
\usepackage{microtype}
\DisableLigatures[f]{encoding = *, family = * }

% color can be used to apply background shading to table cells only
\usepackage[table]{xcolor}
\usepackage{multirow} %connecting columns in tables
\usepackage{multicol}
\usepackage{longtable}

% array package and thick rules for tables
\usepackage{array}

% create "+" rule type for thick vertical lines
\newcolumntype{+}{!{\vrule width 2pt}}

% create \thickcline for thick horizontal lines of variable length
\newlength\savedwidth
\newcommand\thickcline[1]{%
  \noalign{\global\savedwidth\arrayrulewidth\global\arrayrulewidth 2pt}%
  \cline{#1}%
  \noalign{\vskip\arrayrulewidth}%
  \noalign{\global\arrayrulewidth\savedwidth}%
}

% \thickhline command for thick horizontal lines that span the table
\newcommand\thickhline{\noalign{\global\savedwidth\arrayrulewidth\global\arrayrulewidth 2pt}%
\hline
\noalign{\global\arrayrulewidth\savedwidth}}


% Remove comment for double spacing
%\usepackage{setspace} 
%\doublespacing

% Text layout
\raggedright
\setlength{\parindent}{0.5cm}
\textwidth 5.25in 
\textheight 8.75in

% Bold the 'Figure #' in the caption and separate it from the title/caption with a period
% Captions will be left justified
\usepackage[aboveskip=1pt,labelfont=bf,labelsep=period,justification=raggedright,singlelinecheck=off]{caption}
\renewcommand{\figurename}{Fig}

% Use the PLoS provided BiBTeX style
\bibliographystyle{plos2015}

% Remove brackets from numbering in List of References
\makeatletter
\renewcommand{\@biblabel}[1]{\quad#1.}
\makeatother

% Leave date blank
\date{}

% Header and Footer with logo
\usepackage{lastpage,fancyhdr,graphicx}
\usepackage{epstopdf}
\pagestyle{myheadings}
\pagestyle{fancy}
\fancyhf{}
\setlength{\headheight}{27.023pt}
\lhead{\includegraphics[width=2.0in]{PLOS-submission.eps}}
\rfoot{\thepage/\pageref{LastPage}}
\renewcommand{\footrule}{\hrule height 2pt \vspace{2mm}}
\fancyheadoffset[L]{2.25in}
\fancyfootoffset[L]{2.25in}
\lfoot{\sf PLOS}

%% Include all macros below

\newcommand{\lorem}{{\bf LOREM}}
\newcommand{\ipsum}{{\bf IPSUM}}

%% END MACROS SECTION


\begin{document}
\vspace*{0.2in}

% Title must be 250 characters or less.
\begin{flushleft}
{\Large
\textbf\newline{Eco-evolutionary Dynamics on the Invasion Front Drives the Co-existence of Target Site and Quantitative Resistance} % Please use "title case" (capitalize all terms in the title except conjunctions, prepositions, and articles).
}
\newline
% Insert author names, affiliations and corresponding author email (do not include titles, positions, or degrees).
\\
Shaun Coutts\textsuperscript{1*},
Rob Freckelton\textsuperscript{1},
Helen Hicks\textsuperscript{1},
Paul Neve\textsuperscript{2},
Dylan Childs\textsuperscript{1},
\\
\bigskip
\textbf{1} Animal and Plant Sciences, University of Sheffield, Sheffield, UK
\\
\textbf{2} Affiliation Dept/Program/Center, Institution Name, City, State, Country
\\
\bigskip

% Insert additional author notes using the symbols described below. Insert symbol callouts after author names as necessary.
% 
% Remove or comment out the author notes below if they aren't used.
%
% Primary Equal Contribution Note
%\Yinyang These authors contributed equally to this work.

% Additional Equal Contribution Note
% Also use this double-dagger symbol for special authorship notes, such as senior authorship.
%\ddag These authors also contributed equally to this work.

% Current address notes
%\textcurrency Current Address: Dept/Program/Center, Institution Name, City, State, Country % change symbol to "\textcurrency a" if more than one current address note
% \textcurrency b Insert second current address 
% \textcurrency c Insert third current address

% Deceased author note
%\dag Deceased

% Group/Consortium Author Note
%\textpilcrow Membership list can be found in the Acknowledgments section.

% Use the asterisk to denote corresponding authorship and provide email address in note below.
* shaun.coutts@gmail.com

\end{flushleft}
% Please keep the abstract below 300 words
\section*{Abstract}
evolving resistance threatens the use of important antibiotics, pesticides and herbicides (collectively xenobiotics). Xenobiotic resistance is a complex trait, controlled by a highly variable numebr of genes. At one extreme target site resistance is conferred by a single gene and can give very high levels of resistance with few life history costs. At the other extreme polygenic, quantitative resistance is controlled by small additive effects of many genes and tends to give lower levels of resistance and incur some demographic costs. There is also growing evidence that target site and quantitative resistance exist within the same population. This has important implications for understanding how complex traits evolve since high efficacy, low cost target site resistance should out compete the less effective more costly quantitative resistance. There are also important implications for the management of resistance as strategies that slow the evolution of target site resistance may accelerate the evolution of quantitative resistance. We model the evolution of resistance as a complex, duel architecture trait and apply this model to the economically important weed, \textit{Alopecurus myosuroides} (Huds.). We find that quantitative resistance can slow the spread of target site resistance into a population when demographic costs for quantitative resistance are relatively low. We also show that the ecological context into which target site resistance is invading affects where and when both target site and quantitative resistance develop in the same population.      

% Please keep the Author Summary between 150 and 200 words
\section*{Author Summary}
TO BE DONE

\linenumbers

% Use "Eq" instead of "Equation" for equation citations.
\section*{Introduction}
There is currently a global resistance crisis \cite{Serv2013, Ross2014} and important antibiotics, pesticides and herbicides (collectively xenobiotics) are losing their efficacy \cite{Palu2001}. Together, these chemical tools are a crucial part of the world's food productions system \cite{Duke2012}, and are used worldwide to control life threatening infectious diseases. However, evolving resistance threatens their continued use \cite{Barb2011, Nkya2013}. Economic and regulatory conditions mean that bringing new, safe and effective compounds to market is time consuming and expensive \cite{Duke2012}. In addition, the useful life of new xenobiotics can be as short as three years if their use is not well managed \cite{Palu2001, Duke2012}. To address this crisis and reduce its impact we need to understand the factors that promote or constrain the evolution of resistance. 

Xenobiotic resistance is a complex, multi-architecture trait (genetic architecture \textit{sensu} \cite{Deba2015}, the number and effect size of loci contributing to a trait) \cite{Warw1991, Neve2007, Dely2013, Bauc2016}, complicating efforts to understand and manage its evolution. The best understood form of resistance is target site resistance \cite{Warw1991, Neve2007, Dely2013, Serv2013}, which has a simple architecture, often controlled by mutation at a single locus \cite{Bauc2016}. This mutation changes the binding site of the xenobiotic so that it can no longer bind, but the protein retains its original function \cite{Dely2013, Bauc2016}. As a result target site resistance often confers very high levels of resistance, and incurs few life history costs (i.e. reductions in survival, growth and/or fecundity)\cite{Warw1994, Vila2005, Bauc2016}. At the other extreme polygenic, quantitative resistance is conferred by small additive effects of many genes \cite{Land1989, Mack2009, Dely2013, Rajo2013}. Quantitative resistance is usually achieved by metabolizing the xenobiotic, or transporting it away from its binding site \cite{Dely2013, Bauc2016} and tends to confer lower levels of resistance and incur fitness costs \cite{Vila2005, Bauc2016}. There is also growing evidence that target site and quantitative resistance exist within the same population in both plants \cite{Warw1991, Vila2005, Herr2014, Han2016} and insects \cite{Gard1998, Donn2009, Bing2011, Hend2013, Oake2013}.

The genetic architecture that confers resistance has important implications for the management of resistance. For example using lower doses delays the evolution of target site resistance once present in a population, but may accelerate the evolution of quantitative resistance \cite{Gard1998}. If populations have both target site and quantitative resistance managers may face an impossible situation, needing to simultaneously lower and increase the dose to slow the development resistance. It has also been suggested that populations with both genetic architectures operating are more likely to benefit from evolutionary rescue under harsh conditions (e.g. repeated herbicide application) \cite{Gomu2010}. Thus, it is important to understand how target site resistance and quantitative resistance interact and spread through populations, yet this remains largely unexplored, even theoretically (see \cite{Gomu2010, Deba2015, Yeam2015} for exceptions). 

Because quantitative resistance is conferred by many genes \cite{petit2010, Busi2013} and target site resistance is conferred by a small number of loci with large effect, most mutations generate variation in quantitative resistance. Thus, quantitative resistance should develop before target site resistance \cite{Dely2010newPhy}. If multiple resistance genes (possibly of small effect) which incur a fitness cost have a high frequency in a population (i.e. quantitative resistance is already established), target site resistant individuals may inherit these genes, even if they don't require their benefits \cite{Yeam2015}. As a result target site resistant individuals may pay the demographic cost of quantitative resistance, reducing the advantage of target site resistance and slowing its invasion \cite{Chev2008}. Limited dispersal might facilitate this effect by creating areas with high levels of target site resistance near to areas with high quantitative resistance. The area with high quantitative resistance could act as a source of genes which impose a demographic cost \cite{Dely2010, Yeam2015}. On the other hand clumping of genotypes (due to dispersal limitation) could allow target site resistance to take hold more quickly, since individuals with rare genotypes will be more likely to cross with other rare mutants if closely related individuals exist close to each other (i.e. sibling-sibling crosses and parent-child crosses).

We develop a spatially explicit, density dependent population models of the evolution of resistance as a duel architecture trait. We apply this model to the economically important weed, \textit{Alopecurus myosuroides} (Huds.). Our aim is to generate hypotheses about how the initial conditions of the invasion of a new resistance genotype (defined by both $G$ and $g$) can change how that invasion develops over time. We find that quantitative resistance slows the spread of target site resistance into a population when demographic costs for quantitative resistance are relatively low. We also show that dispersal limitation and the ecological context of an invasion occurs affects where and when both target site and quantitative resistance evolve in the same population. 

\section*{Materials and Methods}
% For figure citations, please use "Fig" instead of "Figure".
\subsection*{model}
We model the invasion of a single locus that confers perfect resistance with no fitness costs (perfect target site resistance) in an annual, diploid plant population that harbours additive genetic variance for an independent, costly, quantitative resistance trait. We begin with a spatially implicit model and then develop spatial model where the population exists on a 1D landscape. We use a discrete time, continuous space, modelling framework that can track the evolution of both continuous (quantitative resistance) and discrete (target site resistance) state variables.     

Our annual projection interval starts at the beginning of the growing season before any seeds have emerged from the seed bank. Once seeds germinate they are either exposed to herbicide or not. Survival under herbicide is conferred by two types of resistance, quantitative and target site. Quantitative resistance is determined by the breeding value $g$. Quantitative resistance is modelled assuming the infinitesimal model of inheritance \cite{Fish1918}. We assume target site resistance is controlled by a single locus, with Mendelian inheritance. We assume no mutations in the target site resistance locus. We denote the resistant allele $R$ and the susceptible allele $r$. Those individuals that survive then flower and spread pollen. Finally, survivors disperse their seeds into the seed bank. See Fig. \ref{fig:schematic} for a description of the life cycle.  

\begin{figure}[!h]
	\centering
	\includegraphics[width=140mm]{/home/shauncoutts/Dropbox/projects/MHR_blackgrass/BG_population_model/writting/figures/spatial_model_fig/life_cycle_schem.pdf}
\caption{\bf Schematic of the spatial population model.} The seed bank has one of three target site resistance genotypes $G \in \{RR, Rr, rr\}$ (where $rr$ indivudals are suceptable) and are distributed over quantitative resistance $g$. The emergence and survival functions are applied to the seed bank distributions to derive the above ground parent distributions. Those parent distributions are mixed to create the new seed distributions. \label{fig:schematic}
\end{figure}

Transitions from one year to the next occur via the seed bank. The model projects the seed bank for target site genotype $G \in \{RR, Rr, rr\}$ at time $t$, $b_G(g, t)$, to the seed bank at time $t+1$. $b_G(g, t)$ is a state distribution function of individuals in the seed bank with genotype $G$, quantitative resistance breeding value $g$. Note that $b_G(g, t)$ is an un-normalized density, and so does not integrate to 1. Because $b_G(g, t)$ is an un-normalized distribution over the breeding value of quantitative resistance, $g$, so are the distributions that arise from $b_G(g, t)$ (i.e. $n_G(g, t)$, $n'_G(g, t)$, $n''_G(g, t)$). When we require a probability density we explicitly normalize. In regards to general notation the continuous variable $g$ is inside brackets, while the discrete target site resistance genotype ($G$) is denoted with a subscript and/or a superscript.    

The number of individuals of target site genotype $G$ that emerge from the seed bank and establish at time $t$ is 
\begin{equation}\label{eq:above_ground}
	n_G(g, t) = b_G(g, t)\phi_e,
\end{equation}
where $\phi_e$ is the probability that a seed germinates. The distribution of surviving individuals is 
\begin{equation}\label{eq:abg_sur}
	n'_G(g, t) = n_G(g, t)s_G(g, h) 
\end{equation}
where survival is a function on the target site genotype, $G$, quantitative resistance, $g$, and herbicide application index $h \in \{0, 1\}$.   
\begin{equation}\label{eq:sur_G}
	s_G(g, h) = \begin{cases} 
		\frac{1}{1 + e^{-s_0}} &\text{~if~} G \in \{RR, Rr\} \\
		\frac{(1 - \varsigma)}{1 + e^{-s_0}} + \frac{\varsigma}{1 + e^{-\left(s_0 - h\left(\xi - \textbf{min}(\xi, \rho g) \right)\right)}} &\text{~otherwise~} 		
	\end{cases} 
\end{equation}  
Where $s_0$ is survival (in logits) without the effect of herbicide, $\varsigma$ is the proportion of the population that is exposed to herbicide, $\xi$ is the reduction of survival (in logits) due to herbicide and $\rho$ is the protective effect a one unit increase in $g$ has against herbicide (in logits).   

The effects of density and the demographic costs of resistance are applied to the distribution of survivors. These effects could affect survival or reproduction, or both. In a discreet time model of an annual plant a reduction in fecundity is equivalent to a reduction in survival, as both affect the number of seeds that enter the seed bank. We define the effective reproductive population ($n''_G(g, t)$) as the population after herbicide application, incorporating the demographic costs of resistance and effects of density. 
\begin{equation}\label{eq:effect_pop}
	n''_G(g, t) = \frac{n'_G(g, t)}{1 + \text{exp}(-(f_c^0 - f_c|g|))}\cdot\frac{1}{1 + f_d\sum_{\forall G} \int_g n'_G(g, t)\text{d}g}
\end{equation} 
where $f_c^0$ is the cost of demographic cost of quantitative resistance when its breeding value $g = 0$ (in logits), $|g|$ is the absolute value of value $g$ and $f_c$ is the cost of resistance in terms of in terms of the logit reduction in effective population size for every one unit increase in the breeding value of $g$. Density dependence is controlled by $f_d$, with $1/f_d$ the population density where individuals start to interfere with each other. 

Our model focuses on a monoecious species, thus $n''_G(g, x, t)$ is both the maternal and paternal parent distributions. We assume that the supply of pollen never limits seed production (i.e. pollen is always very abundant). The probability density function of pollen with quantitative resistance $g$ and target site genotype $G$ is 
\begin{equation}\label{eq:pollen_func}
\gamma_G(g, t) = \frac{n''_G(g, t)} {\sum_{\forall G_p}\int_{g_p} n''_{G_p}(g_p, t) \text{d}g_p}, 
\end{equation}

The distribution over $g$ of seeds produced depends on both the paternal and maternal parent distribution. Thus, we must calculate the distribution of seeds over $g$ for each $G_m \times G_p$ cross, 
\begin{equation}
\label{eq:fec_GG}
	\eta_{G_p}^{G_m}(g, t) = \int_{g_m}\int_{g_p} N(g|0.5 g_m + 0.5 g_p, 2V_A)f_\text{max} n''_{G_m}(g_m, t)\gamma_{G_p}(g_p, t)\text{d}g_p\text{d}g_m
\end{equation}          
$G_p$ and $G_m$ denote the paternal and maternal target site genotypes respectively. Similarly, the maternal and paternal quantitative resistance breeding value are denoted $g_m$ and $g_p$. The offspring produced by every pair of $g_m$:$g_p$ values are assumed to be normally distributed with a mean of $0.5g_m + 0.5g_p$ and standard deviation of $2V_A$ (twice the additive variance) \cite{Ture1994}. $f_\text{max}$ is the maximum possible number of seeds per individual, when both density and quantitative resistance are 0.

To get the distribution of seeds over $g$, at each location, for each target site genotype, $G$, $f_{G}(g, t)$, we must sum the distributions of seeds from each $G_m \times G_p$ cross, weighting the distributions by the proportion of seeds of genotype $G$ each cross will produce. 
\begin{subequations}
\label{eq:fec_G}
\begin{align}
	f_{RR}(g, t) &= \eta_{RR}^{RR}(g, t) + \eta_{RR}^{Rr}(g, t)0.5 + \eta_{Rr}^{RR}(g, t)0.5 + \eta_{Rr}^{Rr}(g, t)0.25\\
\begin{split}
	f_{Rr}(g, t) &= \eta_{RR}^{Rr}(g, t)0.5 + \eta_{Rr}^{RR}(g, t)0.5 + \eta_{RR}^{rr}(g, t) + \eta_{Rr}^{Rr}(g, t)0.5 + \eta_{Rr}^{rr}(g, t)0.5 +\\
	 &~~~~\eta_{rr}^{RR}(g, t) + \eta_{rr}^{Rr}(g, t)0.5
\end{split}\\
	f_{rr}(g, t) &= \eta_{rr}^{rr}(g, t) + \eta_{rr}^{Rr}(g, t)0.5 + \eta_{Rr}^{rr}(g, t)0.5 + \eta_{Rr}^{Rr}(g, t)0.25
\end{align}  
\end{subequations}  

We close the life cycle by adding the seeds produced during time step $t$, to the seed bank (Eq. \ref{eq:above_ground}) so that 
\begin{equation}
	b_G(g, t + 1) = b_G(g, t)(1 - \phi_e)\phi_b + f_G(g, t).  
\end{equation}
Where $(1 - \phi_e)$ is the proportion of seeds that did not germinate $\phi_b$ is the probability that a seed in the seed bank, $b_G(g, t)$, survives one year. 

The model was implemented in Julia version 0.5.0, and code for the model and plotting is available in \nameref{S1_File}.
% Place figure captions after the first paragraph in which they are cited.
%\begin{figure}[!h]
%\caption{{\bf Bold the figure title.}
%Figure caption text here, please use this space for the figure panel descriptions instead of using subfigure commands. A: Lorem ipsum dolor sit amet. B: Consectetur adipiscing elit.}
%\label{fig1}
%\end{figure}

\subsection*{Parametrization}
of \textit{A. myosuroides}
of \textit{A. myosuroides} 
\textit{A. myosuroides} is an annual, diploid, out-crossing grass. 

Several of the population model parameters, particularly those relating to the quantitative genetic selection model, are unknown for our study system. We used field-derived estimates of population density to calibrate the model. While we cannot directly parametrize the model with this data, we could use it to calibrate the model and constrain sets of parameter values to those that predicted population sizes in the range of observed population sizes. We outline this procedure in \nameref{S2_Appendix}. Estimates and sources for each parameter are given in Table \ref{tab:parameters}.          

\begin{table}[!ht]
\begin{adjustwidth}{-2.25in}{0in} % Comment out/remove adjustwidth environment if table fits in text column.
\centering
\caption{
{\bf Model parameters with range used in parameter filtering (see \nameref{S2_Appendix}), etimated value, brief description and source}}
\begin{tabular}{|l+l|l|p{9.5cm}|p{2.5cm}|}
\hline
		{\bf parameter} & {\bf range} & {\bf estimate} & {\bf description} & {\bf source}\\
 \thickhline
 &\multicolumn{4}{l|}{{\it Population model}}\\ \hline
	$\phi_b$ & 0.22 -- 0.79 & 0.42 & seed survival probability & \cite{Thom1997}\\ \hline
	$\phi_e$ & 0.45 -- 0.6 & 0.52 & germination probability & \cite{Colb2006}\\ \hline	
	$f_\text{max}$ & 30 -- 300$^\blacklozenge$ & 45 & seed production (in seeds/plant) of highly susceptible individuals at low densities & \cite{Doyl1986}\\ \hline
	$f_d$ & 0 -- 0.15$^\dag$ & 0.004 & reciprocal of population (1/pop.) at which individuals interfere with each others fecundity & \cite{Doyl1986}\\ \hline 
	$f_c^0$ & 4 -- 10$^\dag$ & no est$^\ddag$  & fecundity in a naive population in logit($f_0$)$f_\text{max}$ & simulation\\ \hline
	$f_c$ & $0.1f_0$ -- $2f_0 ^\dag$ & no est$^\ddag$ & reduction in fecundity (in logits) due to a one unit increase in resistance. Only meaningful in relation to $f_0$ & simulation\\ \hline
	$2V_A$ & fixed & 1 & standard deviation of offspring distribution as $2V_A$ (additive variance) & fixed without loss of generality\\ \hline
	$s_0$ & fixed & 10 & survival in a naive population is logit($s_0$) & fixed, use $\xi$ and $\rho$ to control survival function\\ \hline
	&\multicolumn{4}{l|}{{\it Management effects}}\\ \hline
	$\varsigma$ & 0.5 -- 1$^\dag$ & 0.8$^\blacklozenge$ & proportion of above ground individuals exposed to herbicide & HGCA\\ \hline   		
	$\xi$ & $2s_0$ -- $3s_0^\dag$ & no est$^\ddag$ & reduction in survival (in logits) due to herbicide (only meaningful in relation to $s_0$) & simulation\\ \hline	
	$\rho$ & $0.1\xi$ -- $2\xi$ & no est$^\ddag$ & protection against herbicide (in logits) conferred by a one unit increase in $g$, only meaningful in relation to $\xi$ & simulation\\ \hline
	int$_{Rr}$ & 0.001 -- 0.2 & & initial frequency of resistant target site genotype. All other individuals are assumed to be of genotype rr & \\ \hline
	&\multicolumn{4}{l|}{{\it Dispersal}}\\ \hline
	$\alpha$ & 0.38 -- 0.58$^\dag$ & 0.48 & proportion of seeds in short dispersal kernel & \cite{Colb2001}\\ \hline   
	$\mu_1$ & 0.46 -- 0.7$^\dag$ & 0.58 & distance (in m) at which maximum number of seeds are found in short seed dispersal kernel & \cite{Colb2001}\\ \hline
	$\mu_2$ & 1.32 -- 1.98$^\dag$ & 1.65 & distance (in m) at which maximum number of seeds are found in long seed dispersal kernel & \cite{Colb2001}\\ \hline
	$\omega_1$ & 0.35 -- 0.53$^\dag$ & 0.44 & proportion of seeds that disperse up to distance $\mu_1$ in short seed dispersal kernel & \cite{Colb2001}\\ \hline
	$\omega_2$ & 0.31 -- 0.47$^\dag$ & 0.39 & proportion of seeds that disperse up to distance $\mu_2$ in long seed dispersal kernel & \cite{Colb2001}\\ \hline
	$a$ & 25.6 --38.4$^\dag$ & 32.3 & scale parameter for pollen dispersal kernel & \cite{Klei2006}\\ \hline
	$c$ & 2.66 -- 3.98$^\dag$ & 3.32 & shape parameter for pollen dispersal kernel & \cite{Klei2006}\\ \hline
\end{tabular}
\begin{flushleft} $\blacklozenge$ sourced from grey literature, unpublished data and expert opinion\\
	$\dag$ range not available from literature, simulation used to find plausible range\\
	$\ddag$ no estimate not available from literature, simulation used to find plausible range
\end{flushleft}
\label{tab:parameters}
\end{adjustwidth}
\end{table}

\subsection*{Sensitivity analysis}
The model calibration resulted in 11,866 parameter combinations that produced realistic densities. We performed a global sensitivity analysis on this set of 11,866 parameter sets to explore how different parameters, and their interactions, affected the behaviour of the model. We followed the approach of \cite{Cout2014} and fitted Boosted Regression Trees (BRTs) to find relationships between the parameter values and the model behaviours of interest. We focused on three aspects of the models behaviour: the speed at which target site resistance establishes in the population, how much resistance was derived from quantitative resistance, and the rate of population spread. We used the frequency of alleles that are $R$ at final time step $T$ (\%$R_T$) as a metric of target site resistance spread, the survival of target site susceptible individuals under herbicide ($Sur_{rr}$) at the highest mean $g$ achieved as a metric of how important quantitative resistance was, and mean spread rate as a metric of population spread. For a detailed explanation of the sensitivity analysis see \nameref{S2_Appendix}.

\subsection*{Simulation experiments}
Because we assume no mutation in the target site resistance locus, target site resistance can only develop in a population the target site resistant allele needs to be introduced from a population where it has already developed through mutation. It is also possible to import high levels of quantitative resistance. To explore how the composition of the source and receiving populations influence the rate of evolution of resistance in the latter we created two general scenarios: a natural spread scenario, where target site resistance starts at one edge of the landscape, and is allowed to invade into rest of the landscape (Fig. \ref{fig:exper}A); and a transplant scenario (Fig. \ref{fig:exper}B) where a small number of seeds are deposited in the center of a receiving landscape. The natural spread scenario replicates spread at a smaller scales, where the target site genotype is invading into a neighbouring field (possibly under different management), or from a field margin. The transplant scenario replicates the situation when a small amount of seeds are transported to a new, distant, area by farm vehicles. 

To asses the behaviour of the model we use \%$R_T$ to measure how frequent target site resistance was at time $T$ and $Sur_{rr}$ to measure the level of quantitative resistance in a population. We are also interested in the demographic advantage experienced by target site resistant individuals. We measure this as the ratio of the average number of seeds produced by a target site resistant individual that emerges from the seed bank, to the average number of seeds produced by a target site susceptible individual that emerges from the seed bank, under herbicide at location $x$ and time $t$.
\begin{subequations}
\label{eq:TSR_adv}
\begin{equation}
	K(x, t) = \frac{\kappa_R(x, t)}{\kappa_r(x, t)}
\end{equation}
\begin{equation}
	\kappa_R(x, t) = \frac{\int_g (n(g, RR, x_m, t) + n(g, Rr, x_m, t)) s(g, RR, x, h_x = 1)\psi(g, x)\text{d}g}{\int_g n(g, RR, x_m, t) + n(g, Rr, x_m, t)\text{d}g}
\end{equation}
\begin{equation}
	\kappa_r(x, t) = \frac{\int_g n(g, rr, x_m, t) s(g, rr, x, h_x = 1) \psi(g, x)\text{d}g}{\int_g n(g, rr, x_m, t)\text{d}g} 
\end{equation}
\end{subequations}
$K(x, t)$ incorporates the effect of $g$ on both fecundity and survival. When $k = 1$ both target site resistant and susceptible individuals produce the same average number of seeds over their life span. When $K > 1$ for every seed an average target site susceptible individual produces, target site resistant individuals produce $K$ seeds on average.   

Unless otherwise stated in these simulation experiments $f_0 = 4$, $s0 = 10$, $\rho = 1.5$, $f_r = 0.45$ and $\zeta = 14$. With these values an $rr$ individual with quantitative resistance $g = 5$ will have survival of 0.97 (similar to target site resistance survival), but produce $f_{max} 0.85$ seeds and pollen (assuming density is low).

\newpage
\begin{figure}[!h] 
	\includegraphics[height=100mm]{/home/shauncoutts/Dropbox/projects/MHR_blackgrass/BG_population_model/writting/figures/experiments_schematic.pdf}
\caption{\bf Natural spread and transplant simulation experiments.} {\bf A)} Natural spread experiments where the source and receiving populations exist on the same landscape. Source populations started with 1\% of alleles being target site resistant (i.e. $R$), while receiving populations started with no target site resistance. The whole landscape was exposed to herbicide for 150 time steps. {\bf B)} Transplant experiments where 10 seeds are added to the seed bank at the center of the receiving landscape after 20 time steps. The added seeds have one of four combinations of frequency of target site resistance and the amount of quantitative resistance (coloured text in source population plane). The receiving population was in one of three states (empty, exposed to herbicide for 20 time steps, naive to herbicide). After the 20 time step establishment period the entire landscape was exposed to herbicide. In both natural spread and transplant experiments non-empty locations were initialized with 10 $rr$ individuals, with low quantitative resistance ($g = 0$). 
\label{fig:exper}
\end{figure}

% Results and Discussion can be combined.
\section*{Results}
The ecological and genetic background against which the invasion of the resistance genotype took place had an enormous effect on how the two genetic architectures underlying resistance (target site and quantitative resistance) interacted. In both the transplant and natural spread experiments it was only when populations with small amounts of target site resistance expanded into empty landscapes that target site resistance became the dominant mode of resistance. The details of how this develops over time are best shown by examining a run of the natural spread experiment (Fig. \ref{fig:pro_R_natspr}). Under constant herbicide application target site resistance allele quickly becomes prevalent in the population (source region Fig. \ref{fig:pro_R_natspr}a,c). When the receiving region was empty this occurred before the population has completely invaded into the empty area (white regions on the left hand side of the heat maps). Thus, after target site resistance had become established it spread with the population and so became established over the whole landscape (Fig. \ref{fig:pro_R_natspr}a).   

The dynamics of the system are very different when target site resistance alleles are invading into an area that already has \textit{A. myosuroides}. In these cases target site alleles invade very slowly at first (reds and purples stay near the source region early on in Fig. \ref{fig:pro_R_natspr}c). The invasion of target site resistant alleles then accelerates. When the receiving region was exposed to herbicide the population already there developed high levels of quantitative resistance by the time the target site resistant alleles reached that part of the landscape (Fig \ref{fig:TSR_adv}). These high levels of metabolic resistance reduced the selective pressure and slowed the spread of target site resistant alleles. The acceleration occurs because the pollen dispersal kernel is quiet flat at the scales examined in these experiments, and so target site resistant alleles could reach large parts of the receiving landscape. At a certain point the there are enough target site resistant alleles over the whole landscape that their frequency builds quickly across the whole landscape almost simultaneously (creating the curved patterns in Fig. \ref{fig:pro_R_natspr}c and d). Even with this accelerating invasion, the invasion of target site resistant alleles across the whole landscape took over 100 years longer across an empty landscape than when invading across a landscape that already had \textit{A. myosuroides} present (Fig. \ref{fig:pro_R_natspr}). This is a surprising result since the pollen dispersal kernel (which can spread genes only into an existing population) is much flatter than the seed dispersal kernel (which can disperse genes into an empty landscape via seeds).         

\begin{figure}[!h] 
	\includegraphics[height=90mm]{/home/shauncoutts/Dropbox/projects/MHR_blackgrass/BG_population_model/model_output/all_exposed_TSR_and_bySURrr2.pdf}
\caption{\bf Invasion of target site resistant allele, $R$, in the natural spread experiments.} The population over the whole landscape (y-axis) is shown at each time step (x-axis; i.e. each slice is a snapshot of the population). Hue indicates either, the proportion of all alleles at a location and time that are $R$ (\%$R_T$; a,c), or \%$R_T$, multiplied by the survival of target site susceptible individuals under herbicide (b,d). This latter metric shows the time and place where both target site resistance and quantitative resistance exist at high levels. The source population (containing target site resistant alleles) is invading into an empty landscape (a, b), or an area that already has \textit{A. myosuroides} present, but no target site resistance alleles (c,d). 
\label{fig:pro_R_natspr}
\end{figure}

Populations only displayed both high levels of quantitative resistance and high frequencies of target site resistance alleles when target site resistance alleles were invading in to a region that already had \textit{A. myosuroides}. Even then both genetic architectures were only important at the invasion front of the target site resistant alleles(purple region in Fig. \ref{fig:pro_R_natspr}d). This occurred because high quantitative resistance was imported back into the region with high target site resistance via pollen and seeds, increasing quantitative resistance there.        

In our model target site resistant individuals are, by definition, fitter than target site susceptible individuals under herbicide application. Target site resistant individuals have higher survival, and do not have to incur the demographic costs of high quantitative resistance. However, the demographic costs of quantitative resistance ($f_r$) are incurred whether an individual is using quantitative resistance or not. As a result the fitness difference between target site resistant individuals and target site susceptible individuals was small in parts of the landscape where target site resistant individuals were rare (light blue regions of Fig. \ref{fig:TSR_adv}). This occurred because high levels of quantitative resistance built up in these regions. The pollen from these individual swamped the pollen from the target site resistant individuals and meant that their offspring also tend to have high levels of quantitative resistance. Thus, even if quantitative resistance was not useful for target site resistant individuals, they still incurred much of the demographic cost.

\begin{figure}[!h] 
	\includegraphics[height=90mm]{/home/shauncoutts/Dropbox/projects/MHR_blackgrass/BG_population_model/model_output/TSR_adv_ee.pdf}
\caption{\bf Reproductive advantage of target site resistant individuals over target site susceptible individuals at each location at each time ($K$).} Natural spread experiments were run where the whole landscape was exposed to herbicide. Target site resistance alleles either invaded into an empty landscape (a) or into an area that already had \textit{A. myosuroides} present, but no target site resistance alleles (b). At each time ($t$) and location ($x$) the reproductive advantage of target site resistant individuals was calculated ($K(x, t)$, Eq. \ref{eq:TSR_adv}). $K = 1$ means  target site resistant and susceptible individuals have the same reproductive output under herbicide.   
\label{fig:TSR_adv}
\end{figure}

The sensitivity analysis showed that the protective effect of a one unit increase in the quantitative resistance trait $g$ ($\rho$) and the demographic costs of quantitative resistance ($f_f$), had the largest effect on genetic dynamics of the population (\nameref{S2_Appendix}). These parameters are crucially important for when both genetic architectures can co-exist, with co-existence only common at $f_r < 0.8$ and $\rho > 1.25$ and being more common when the  population was invading into and empty landscape (\nameref{S2_Fig}c,d)   

We use transplant experiments to model situations where genes from populations with different management histories, and thus different levels of target site and quantitative resistance, invade into a new population. A situation commonly arising with long distance dispersal of propagules. These experiments show that the characteristics of the source population and receiving area, and the protective effect of quantitative resistance, $\rho$, all influence the simultaneous evolution of both target site and quantitative resistance. 

Regardless of the genetic structure of the introduced seeds, if those seeds were introduced into a landscape where \textit{A. myosuroides} was already present (light and dark blue lines in Fig. \ref{fig:transloc_gpro}) target site and quantitative resistance could only co-exist over a small range of values for $\rho$. Either target site resistance dominated (solid blue lines high and dashed blue lines low) or quantitative resistance dominated (dashed blue lines high, solid blue lines low). This pattern was most pronounced when the receiving population (where target site resistance was absent) had been exposed to herbicide before target site alleles were introduced (dark blue lines). When the receiving population was naive to herbicide (light blue lines) the level of quantitative resistance in the population after 100 years still showed a threshold relationship with $\rho$, but the amount of target site resistance in the population reduced more slowly with increasing $\rho$. Thus, target site and quantitative resistance co-existed over a greater range of values when target site alleles were introduced to a naive population.

When target site resistant alleles were introduced into an empty landscape the genotype of the source population had a large effect on how the invasion unfolded. When the source population had low quantitative resistance and a low frequency of target site resistance alleles target site resistance was always high, and quantitative resistance also became important at high values of $\rho$ (black lines Fig. \ref{fig:transloc_gpro}a). When the source populations had a high frequency of target site resistant alleles and low quantitative resistance target site resistance always dominated (black lines Fig. \ref{fig:transloc_gpro}b). When the source population had low target site resistance and high quantitative resistance, both genetic architectures could co-exist in the population, over a wide range of values for $\rho$, although at very low values of $\rho$ target site resistance dominated, and at very high values of $\rho$ quantitative resistance dominated (black lines Fig. \ref{fig:transloc_gpro}c). When the source population had both high quantitative resistance and high target site resistance, then target site resistant alleles were very prevalent in the population for all the values of $\rho$ we tested. At values of $\rho > 1.5$ the population also had high levels of quantitative resistance (Fig. \ref{fig:transloc_gpro}d).                      

\begin{figure}[!h] 
	\includegraphics[height=90mm]{/home/shauncoutts/Dropbox/projects/MHR_blackgrass/BG_population_model/model_output/effect_g_pro_resist.pdf}
\caption{\bf Effect of $\rho$ (protective effect of $g$) on target site and quantitative resistance in transplant experiments.} All populations had a establishment period of 20 time steps and were then exposed to herbicide for 100 time steps, after which time target site and quantitative resistance were measured. Target site resistance was measured as percentage of all alleles in the population that are $R$ (solid lines). Quantitative resistance was measured as the survival of target site susceptible individuals ($rr$) under herbicide application (dashed lines). We tested source populations with four different combinations of target site and quantitative resistance (a, b, c, d), and three different types of receiving landscape: empty--black lines, \textit{A. myosuroides} already present and naive to herbicide--light blue lines, and \textit{A. myosuroides} already present and exposed to herbicide for 20 time steps--dark blue lines.        
\label{fig:transloc_gpro}
\end{figure}

\section*{Discussion}
We find that the ecological and historical context surrounding the invasion of target site resistance into a new area can have an enormous effect on how that invasion unfolds and which genetic architecture dominates. The invasion of target site resistance alleles was much slower when the invasion was into an area that already contained \textit{A. myosuroides}. In addition the frequency of target site resistance and amount of quantitative resistance changed very quickly over space, and on the target site resistance invasion front both architectures co-existed for long periods of time. This occurred even though target site resistance conferred perfect resistance with no demographic costs, giving target site resistant individuals a competitive advantage over those individuals with only quantitative resistance. When the invasion occurred into an empty landscape its dynamics were determined the genetic architecture of the source population and the of quantitative resistance. Given this variability we might expect to see different architectures predominate in different regions of a landscape. To date there is little evidence to either support or refute this hypothesis because the prevalence and type of resistance has not typically been studied at landscape scales \cite{Dely2013}.

We find two scenarios where co-existence of both target site and quantitative resistance occurred; firstly on the invasion front when target site resistant alleles are invading into and existing population where quantitative resistance has developed. This scenario suggests that if herbicides are applied in a spatial mosaic (a strategy recommended to slow the evolution of resistance \cite{Rex2013}), there may be boarder areas where both architectures co-exist. We also found co-existence of both target site and quantitative resistance across the whole population when the invasion occurred into an empty landscape and the source population had high levels of quantitative resistance (when the protective effect of quantitative resistance was sufficiently high). This suggests co-existence will be most common at the leading edge of an invasion, where expanding nascent population foci are common \cite{Mooy1988}. Both these scenarios make it almost inevitable that both genetic architectures co-exist somewhere in the landscape, although the area may be small.

This has important implications for the management of resistance. The continuous nature of quantitative resistance means that there is always some variation for selection to act on and so it can develop earlier \cite{Dely2010newPhy}. Quantitative resistance can also give cross resistance between compounds \cite{Bauc2016, Neve2007}. While target site resistance can confer high levels of resistance for little demographic cost \cite{Bauc2016}. Places where both architectures co-exist may experience the worst of both worlds, resistance that develops very quickly, with very high levels of resistance to existing compounds \cite{Vera2015}, and cross resistance to novel compounds when they are introduced \cite{Neve2007}. These differences in behaviour also mean different strategies are used to manage target site and quantitative resistance (fore example compound cycling to delay the evolution of target site resistance \cite{Rex2013}) \cite{Gard1998}. When populations have both target site and quantitative resistance we may find ourselves with few good strategies to slow the evolution of resistance \cite{Gard1998}. 

Because we assume target site resistance is both more effective and less costly than quantitative resistance, under constant selection all resistance in the population will eventually be target site resistance. However, quantitative resistance can occur for long periods of time when it has a low demographic cost, and a moderate to strong protective effect. In our model these quantities were fixed parameters. In reality both the protective effect of quantitative resistance, and the demographic costs it incurs will also be under selection. Over time the metabolic pathways and other mechanisms of quantitative resistance that provide the most protection for the least demographic cost, will predominate. This gives reason to believe that quantitative resistance can be highly effective and have a low demographic costs [REF].    

\section*{Conclusion}
Resistance affects food production systems and human health across the global. Our results suggest that quantitative resistance can greatly slow the invasion of target site resistance, and the evolutionary dynamics within nascent population foci on the invasion front can result in a different genetic architectures dominating or co-existing. As a result we should expect the genetic architecture underlying resistance to be heterogeneous at both local and landscape scales. It follows from this that the effectiveness of different resistance management strategies, like compound cycling and very high doses, may also vary over space, as these different strategies exploit weaknesses in different genetic architectures \cite{Gard1998, Rex2013}. Our model is a first step in generating hypothesis on how the different genetic architectures that underlie resistance will interact across space. It is imperative that this work is followed by empirical observations and experimental tests. A logical starting point would be spatially extensive surveys of resistant populations to see where and when target site and quantitative resistance dominate or co-exist.

\section*{Supporting Information}

% Include only the SI item label in the paragraph heading. Use the \nameref{label} command to cite SI items in the text.
\paragraph*{S1 Fig.}
\label{S1_Fig}
{\bf Heat maps of survival of target site susceptible plants under herbicide}, a measure of the importance of quantitative resistance. These plots show the development of quantitative resistance over time and space, under three patterns of herbicide use.

\paragraph*{S2 Fig.}
\label{S2_Fig}
{\bf The frequency of target site resistant alleles (\%R) and the amount of target site and quantitative resistance coexistence, over $f_r$ and $\rho$ parameter space.} 

\paragraph*{S1 File.}
\label{S1_File}
{\bf Julia implementation of model and plotting code}  

\paragraph*{S1 Appendix.}
\label{S1_Appendix}
{\bf Target site mixing function.} 
Expanded explanation and worked example of the target site mixing function $Q(G, G_m, G_p)$.

\paragraph*{S1 Appendix.}
\label{S2_Appendix}
{\bf Parameter sanity check and sensitivity analysis} 

\section*{Acknowledgments}
We thank some people for some things

\nolinenumbers

% Either type in your references using
% \begin{thebibliography}{}
% \bibitem{}
% Text
% \end{thebibliography}
%
% or
% % Compile your BiBTeX database using our plos2015.bst
% style file and paste the contents of your .bbl file
% here.
% 
%\begin{thebibliography}{10}
%
%\bibitem{bib1}

%Conant GC, Wolfe KH.
%\newblock {{T}urning a hobby into a job: how duplicated genes find new
%  functions}.
%\newblock Nat Rev Genet. 2008 Dec;9(12):938--950.

%\end{thebibliography}

%REMEMBER TO FIX UP THESE CITATIONS AND USE THEIR FORMATTING ONCE I HAVE A MORE COMPLEATE VERSION.

%\bibliographystyle{/home/shauncoutts/Dropbox/shauns_paper/referencing/bes} 
\bibliography{/home/shauncoutts/Dropbox/shauns_paper/referencing/refs}


\end{document}

