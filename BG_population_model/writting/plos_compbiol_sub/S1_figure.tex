\documentclass[12pt,a4paper]{article}
\usepackage{setspace, graphicx, lineno, color, float}
\usepackage{lscape}
\usepackage[utf8]{inputenc}
\usepackage{amsmath}
\usepackage{amsfonts}
\usepackage{amssymb}
\usepackage[retainorgcmds]{IEEEtrantools}
%page set up
\usepackage[left=2.5cm,top=2.5cm,right=2.5cm,bottom=2.5cm,nohead]{geometry}
\usepackage[aboveskip=1pt,labelfont=bf,labelsep=period,justification=raggedright,singlelinecheck=off]{caption}
\renewcommand{\figurename}{Fig}

\begin{document}

\begin{figure}[!h] 
	\includegraphics[height=110mm]{/home/shauncoutts/Dropbox/projects/MHR_blackgrass/BG_population_model/model_output/sur_rr_time_space.pdf}
\caption{\bf Survival of target site susceptible plants ($G = rr$) under herbicide in the natural spread experiments.} The population over the whole landscape (y-axis) is shown at each time step (x-axis; i.e. each slice is a snapshot of the population). Colour intensity shows population density at each location at each time step, with empty locations being white. Hue indicates survival of target site susceptible individuals under herbicide (a measure of quantitative resistance). In a), b) and c) the source population (containing target site alleles) is invading into an empty landscape, while in d), e) and f) the source population is invading into an area that already has \textit{A. myosuroides} present, but no target site resistance alleles. Each column shows a different pattern of herbicide application. We do not show the combination where both source and receiving populations are naive to herbicide as there is no selective pressure, and so development of resistance under this herbicide regime.    
\end{figure}

\end{document}

