\documentclass[12pt, a4paper]{article}

\usepackage{setspace, graphicx, lineno, caption, color, float}
\usepackage{amsmath}
\usepackage{amsfonts}
\usepackage{amssymb}
\usepackage{amsthm}
\usepackage{amstext} %to enter text in mathematical formulae
\usepackage[retainorgcmds]{IEEEtrantools}
\usepackage{natbib}
%\usepackage{url, hyperref, makeidx, fancyhdr, booktabs, palatino}
%\usepackage{euscript} %EuScript, command: \EuScript, for letters in Euler script
%\usepackage{paralist} %listing (i), (ii), etc.
\usepackage{rotating} %rotating text
\usepackage{multirow} %connecting columns in tables
\usepackage{multicol}
%image stuff
%\usepackage{epstopdf}
%\usepackage{cancel}
%\usepackage[ngerman, english]{babel} %for different languages in one document
%\usepackage[utf8]{inputenc}
%\hypersetup{colorlinks=true, linkcolor=blue}
%page set up
\usepackage[left=2.5cm,top=2.5cm,right=2.5cm, bottom=2.5cm,nohead]{geometry}
\doublespacing
%paragraph formatting
\setlength{\parskip}{12pt}
\setlength{\parindent}{0cm}
\begin{document}
%Make a title page
\title{Black grass IPM}
\author{Shaun R. Coutts$^\dag$, Dylan Childs$^\dag$, Rob Freckleton$^\ddag$}
\maketitle
\section{non-spatial IPM}
We develop a simple model of herbicide resistance in black grass, where resistance is assumed to be a quantitative trait. We model the management of black grass under the evolution of herbicide resistance, where control options are chemical control, that leads to herbicide resistance with prolonged use, and above and below ground cultural control, which is more costly but does not lead to evolved resistance. We find the optimal combination of control options over a finite time horizon, aiming either to maximize net present value (i.e. time discounted farm income) or minimize resistance with a fixed budget (where budget includes opportunity cost). We use two optimization tools, a greedy forward search of the decision space for problems with small time horizon, and a genetic algorithm to find near optimal sequences of actions over larger time horizons.

There are two main questions, firstly, under what conditions would a profit maximizing manager act in a way that avoided selecting for herbicide resistance. Secondly is there a set of actions that can meaningfully reduce herbicide resistance for a reasonable cost. This might apply in cases where profit is not the main motivator, such as control of invasive species on public lands (e.g. national parks and road sides).                  

\subsection{overall model}
We model the population using a yearly time step and assume all seeds that emerge do so from the seed bank. The number with resistance score $g$ at time $t$, $n(g, t)$, is taken just post emergence before any mortality has acted on the population. Thus, $n(g, t)$ is simply the number of individuals that emerge from the seed bank    
\begin{equation}\label{eq:estab}
	n(g, t + 1) = b(g, d = 1, t)\phi_e + b(g, d = 2, t)\phi_e\varphi_e		
\end{equation} 
Where $b(g, d, t)$ is the number of seeds in the seed bank at depth $d \in \{1, 2\}$, with resistance score $g$ at time $t$. $\phi_e$ is the germination probability and $\varphi_e \in [0, 1]$. Both $n(g, t)$ and $b(g, d, t)$ are distributions across $g$.  

\subsection{genetic model}
We assume that herbicide resistance is affected by lots of genes that might control toxin metabolism or toxin transport, each with a small effect, thus we model resistance as a quantitative trait. We assume that black grass are diploid, and are obligate out-crossers.   

\subsection{seed bank and fecundity}
We assume that all seeds germinate out of the seed bank at the start of the time step. We model the number of seeds in the seed bank with resistance $g$ at depth $d$, as the seeds already in the seed bank at depth $d$ that don't germinate and do survive one time step (first term in Eq. \ref{eq:seedbank}) plus the number of seeds added to the top level seed bank at the end of the previous time step, $f(g, t -1)$. We also move seeds between depth levels in response to plowing.
\begin{subequations}\label{eq:seedbank}
	\begin{equation}\label{eq:seedbank_top}
	\begin{split}
		b(g, d = 1, t) = b&(g, d = 1, t - 1)(1 - \phi_e)\phi_b + f(g, t - 1) +\\ 
		&b(g, d = 2, t - 1)p(2 \rightarrow 1|a_{b,t}) - b(g, d = 1, t - 1)p(1 \rightarrow 2|a_{b,t})   
	\end{split}
	\end{equation}
	\begin{equation}\label{eq:seedbank_bottom}
	\begin{split}	
		b(g, d = 2, t) = b&(g, d = 2, t - 1)(1 - \phi_e\varphi_e)\phi_b +\\ 
		&b(g, d = 1, t - 1)p(1 \rightarrow 2|a_{b,t}) - b(g, d = 2, t - 1)p(2 \rightarrow 1|a_{b,t})
	\end{split}
	\end{equation}
\end{subequations} 
Where $b(g, d, t - 1)$ is the number of seeds at depth $d$ in the seed bank resistance score $g$ in the previous time step. $\phi_e(d)$ is the proportion of seeds that germinate and is a function of seed depth
\begin{equation}
	\phi_e(d) = \begin{cases}
	d = 1~\text{:}~\phi_e \\
	d = 2~\text{:}~\phi_e \varphi_e  	
	\end{cases}
\end{equation}
where $\phi_e$ is the germination rate from the top of the soil profile and $\varphi_e \in [0, 1]$ is the proportional reduction in germination rate for seeds deeper in the soil profile, at depth $d = 2$. $d'$ is the depth class not being assessed (i.e. $d \neq d'$) and $p(d' \rightarrow d|a_{b,t})$ is the proportion of seeds that move from depth class $d'$ to depth class $d$ given action to manage the seed bank $a_b \in {\text{plow}, \text{no\_plow}}$ is taken in time $t$. 
\begin{equation}
	p(d' \rightarrow d|a_{b,t}) = \begin{cases}
		a_{b,t} = \text{plow : } \varrho \\
		a_{b,t} = \text{no\_plow : } 0
	\end{cases}
\end{equation}   

 $\phi_b$ is the probability that a seed survives. The distribution of new seeds over resistance $g$,  $f(g, t)$, is modelled as the joint distribution of survivors after management action $a$ Eq. \ref{eq:fecund}. This forms the link between the genetic model of evolution, the population model and the management actions.      
\begin{subequations}
\label{eq:fecund}
\begin{align}
	\displaystyle
	\label{eq:fecund_main}
	f(g, t) &= \\
	\label{eq:fecund_maternal}
	Q(g_i, g') &= 
	\begin{cases}
		0.5 ~\text{if}~ g_i \in g' \\
		\mu ~\text{otherwise}
	\end{cases}\\
	\label{eq:fecund_pollen}
	q(i, t) &= \frac{\displaystyle\sum_{\forall g^* \in G} \eta(i, g^*) \int \text{d}z~n(g^*, z, t)}{\displaystyle \sum_{\forall g^* \in G} \int \text{d}z~n(g^*, z, t)}\\
	\label{eq:pollen_freq}
	\eta(i, g^*) &= 
	\begin{cases}
	0.5~\text{if}~g_{k \neq i} \in g^* \\
	\mu ~\text{otherwise}
	\end{cases}    
\end{align} 
\end{subequations}
The seeds produced by individuals with TSR genotype $g$ that survive until the end of time $t$ are assumed to have NTSR scores, $z$ that follow the normal distribution $Z(t)$ described in Eq. \ref{eq:MR_dist} (total number of seeds multiplied by $Z(t)$ in Eq. \ref{eq:fecund_main}). Because black grass is an out-crossing species individuals from other genotypes ($g'$) can contribute gene copies to the genotype being evaluated ($g$), thus in Eq. \ref{eq:fecund_main} the first summation term sums the seeds of TSR genotype $g$ from all possible TSR genotype parents ($\forall g' \in G$). The integration term in Eq. \ref{eq:fecund_main} gives the number of seeds produced by all individuals of genotype $g'$, where $s(g', z, t)$ is the probability of survival until flowering for an individual of TSR genotype $g'$, with NTSR score $z$ if they germinated at the start of time $t$, and is calculated in Eq. \ref{eq:survival}. $\psi(z)$ is the number of seeds produced by an individual with NTSR score $z$, modelled as a decreasing function of $z$.
\begin{equation}\label{eq:fecund_seed_num}
	\psi(z) = \frac{\psi_\text{max}}{1 + e^{-\alpha_f(z - z_{50}^f)}}
\end{equation}         
where $\psi_\text{max}$ is the maximum number of seeds an individual can produce, $e$ is Euler's constant, $\alpha_f$ is a shape parameter that affects how quickly increasing MR score, $z$, reduces fecundity and $z_{50}^f$ is the $z$ score where fecundity is 50\% of $\psi_\text{max}$. The second summation term across gene copies $i$ gives the probability seeds with tuple $g$ will be produced by an individual of genotype $g'$. Genes copies, $g_i$, have two sources, the individual producing the seed provides half the copies of a gene, while pollen provides the other half. In Eq. \ref{eq:fecund_main} these two processes are modelled with the functions $Q(g_i, g)$, which contributes gene versions ($g_i$) from the mother, and $q(i, t)$, which contributes gene versions from pollen. $Q(g_i, g)$ is a function that returns 0.5 if the value held by copy $g_i$ in the target genotype $g$ is in the genotype being evaluated $g'$, and returns $\mu$ otherwise. $\mu$ is the probability that gene version $g'_i$ mutates gene $g_i$. This function returns 0.5 since the mother can only supply one out of two copies for each gene. The second copy of the gene is supplied from the pollen using function $q(i, t)$. We assume that the probability of getting gene version $g_i$ required to complete the tuple $g$ is the proportion of all pollen which contains that gene version. We assume that the amount of pollen produced with TSR gene version $g_i$ is proportional to the proportion of individuals with that gene copy. Thus, the denominator of Eq. \ref{eq:fecund_pollen} is proportional to the total amount of pollen for all genotypes. The numerator of Eq. \ref{eq:fecund_pollen} is proportional to the amount of pollen that contains the gene version in the target genotype $g$, that is not supplied by the mother. It is function $\eta(i, g^*)$ that determines if this gene version will be produced by the genotype being evaluated for pollen production, $g^*$ (Eq. \ref{eq:pollen_freq}). $\eta(i, g^*)$ returns 0.5 if the version of the gene not provided by copy $g_i$ (i.e. $g_{k \neq i}$ , where $gi$ is the copy provided by the mother) is a member of the tuple $g^*$, the genotype being evaluated for pollen production.

\subsection{survival}   
We assume that probability of survival to flowering and seed set, $s(g, z, t)$ is influenced by TSR genotype $g$, NTSR score $z$, along with herbicide application and density at time $t$. It is at this point that herbicide application separates the different genotypes and drives the spread of herbicide resistant.     
\begin{equation}\label{eq:survival}
	s(g, z, t) = \gamma(t)\phi_s(h(t), z, g) 
\end{equation}

Where $\phi_s(h(t), z, g)$ is the density independent survival probability for an individual with genotype $g$ given the herbicide application in time step $t$, $h(t)$, and is a distribution across $z$. $\gamma(t)$ is a density dependent term that reduces establishment and survival above some density threshold.  

Survival probability, $s(h(t), z, g)$, must be bound between 0 and 1, thus we model it as a logistic function of genotype     
\begin{equation}\label{eq:herb_surv}
	\phi_s(h(t), z, g) =	\phi_\text{min}^{h(t)} + \frac{\phi_\text{max}^z - \phi_\text{min}^{h(t)}}{1 + e^{R(z, g) - R_{50}^s}}	
\end{equation}
where $\phi_\text{min}^{h(t)}$ is the survival rate for a completely susceptible individual and is a decreasing function of herbicide dose in time $t$, $h(t)$ 
\begin{equation}\label{eq:dose_response}
	\phi_\text{min}^{h(t)} = \frac{\phi_\text{max}^z e^{-\xi h(t)}}{1 + \phi_\text{max}^z(e^{-\xi h(t)} - 1)}
\end{equation}
where $\xi$ is a shape parameter that determines how increasing herbicide dose decreases survival for susceptible individuals and $h_{50}$ is the herbicide dose where $\phi_\text{min}^{h(t)}$ is half of $\phi_\text{max}^z$. $\phi_\text{max}^z$ is the survival rate for a maximally resistant individual and is a decreasing function of NTSR score $z$. It can be thought of as the survival cost associated with NTSR and is calculated as 
\begin{equation}\label{eq:surv_cost}
	\phi_\text{max}^z = \frac{\phi_s^0}{1 + e^{-\alpha_s(z - z_{50}^s)}} 
\end{equation}
where $\phi_s^0$ is survival rate for a completely susceptible individual when $h(t) = 0$, $\alpha_s$ is a parameter that controls how quickly increasing NTSR decreases maximum possible survival (cost parameter) and $z_{50}^s$ is the value of $z$ where $\phi_\text{max}^z$ is half $\phi_s^0$.       

$R(z, g)$ is a function that combines the NTSR resistance score $z$ with the TSR genotype $g$ to give an overall resistance score. 
\begin{equation}\label{eq:resist}
	R(z, g) = rz + \textbf{max}(\rho(g_1), \rho(g_2)) + \text{N}(0, \sigma_e)
\end{equation}
where $r$ is a coefficient that increases survival under herbicide application so that higher $z$ scores lead to higher survival probabilities. $\text{N}(0, \sigma_e)$ is a normal distribution with a mean of 0 and a standard deviation of $\sigma_e$. $\sigma_e$ can be thought of as environmental noise and non-heritable factors that increase or decrease resistance to herbicide. This noise term makes $R(z, g, h(t))$ a distribution of values for every value of $z$. Putting the environmental noise inside the function $\phi_s(h(t), z g)$ implies that different seeds experience different levels of environmental impact on resistance. It is also possible that climatic conditions affect the effectiveness of herbicide so that all individuals experience the same increase or decrease in herbicide effectiveness. This could be done in numerical simulations by drawing a random variable for a normal distribution once at the start of the time step and add this number to Eq. \ref{eq:resist}. $\rho(g_i)$ is the amount of herbicide resistance conferred by having the version of gene $g$ held in copy $g_i$. We assume $\rho(v_0) = 0$ and for versions that confer resistance we assume $\rho(v_r) > 0$. Thus, the $\textbf{max}(\cdot)$ function assumes that the resistance version of the gene is dominant. We could use other forms to make different assumptions. For example we could use $\textbf{min}(\rho(g_1), \rho(g_2))$ to make resistant genes receive, or $\rho(g_1) + \rho(g_2)$ to make the copies additive in their effect. 

Without a density dependent term any increasing population would continue growing until infinitely large. We model this density dependence with the term 
\begin{equation}\label{eq:density_dependence}
	\gamma(t) =
		\begin{cases}
			\frac{M}{\sum_{\forall g}\int \text{d}z~ s(g, z, t)b(g, z, t)\phi_e} &\text{if}~ \sum_{\forall g}\int \text{d}z~ s(g, z, t)b(g, z, t)\phi_e > M \\
			1 &\text{if}~ \sum_{\forall g}\int \text{d}z~ s(g, z, t)b(g, z, t)\phi_e \leq M  
		\end{cases}
\end{equation}

where $\gamma(t)$ reduces establishment and survival probability in response to the total number of individuals in the population which are expected to establish at time $t$. $s(g, z, t)$ and $b(g, z, t)$ are defined in Eq.'s \ref{eq:survival} and \ref{eq:seedbank} respectively, $\phi_e$ is defined in \ref{eq:estab}. $M$ is the number of individuals that can exist above ground in the population at any one time. 

\bibliographystyle{/home/shauncoutts/Dropbox/shauns_paper/referencing/bes} 
\bibliography{/home/shauncoutts/Dropbox/shauns_paper/referencing/refs}

\end{document}