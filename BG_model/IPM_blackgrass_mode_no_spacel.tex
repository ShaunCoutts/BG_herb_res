\documentclass[12pt, a4paper]{article}

\usepackage{setspace, graphicx, lineno, caption, color, float}
\usepackage{amsmath}
\usepackage{amsfonts}
\usepackage{amssymb}
\usepackage{amsthm}
\usepackage{amstext} %to enter text in mathematical formulae
\usepackage[retainorgcmds]{IEEEtrantools}
\usepackage{natbib}
%\usepackage{url, hyperref, makeidx, fancyhdr, booktabs, palatino}
%\usepackage{euscript} %EuScript, command: \EuScript, for letters in Euler script
%\usepackage{paralist} %listing (i), (ii), etc.
\usepackage{rotating} %rotating text
\usepackage{multirow} %connecting columns in tables
\usepackage{multicol}
%image stuff
%\usepackage{epstopdf}
%\usepackage{cancel}
%\usepackage[ngerman, english]{babel} %for different languages in one document
%\usepackage[utf8]{inputenc}
%\hypersetup{colorlinks=true, linkcolor=blue}
%page set up
\usepackage[left=2.5cm,top=2.5cm,right=2.5cm, bottom=2.5cm,nohead]{geometry}
\doublespacing
%paragraph formatting
\setlength{\parskip}{12pt}
\setlength{\parindent}{0cm}
\begin{document}
%Make a title page
\title{Black grass IPM}
\author{Shaun R. Coutts$^\dag$, Dylan Childs$^\dag$, Rob Freckleton$^\ddag$}
\maketitle
\section{non-spatial IPM}
We develop a simple model of two genotypes, target site resistant (TSR) individuals and metabolically resistant (MRS) individuals. In TSR plants resistance is very high (i.e. the lethal dose is far higher than any economically feasible herbicide application rate) and incurs no life-history costs. In MR plants resistance is lower and incurs life history costs. The goal of this model is to test how the strength and speed of MR affects the speed of TSR, and how both types of resistance affect the growth of a weeds population over time.

need a MR population number and a TSR population number, probability that a MR individual creates a TSR offspring, the number of TSR a TSR individual will produce, the number of MR offspring a TSR individual will produce and the number of MR individuals a MR individual will produce. Also need the mean and variance in LD50 for the next generation of MR individuals. Also 1 or 2 different types of herbicide with MR individuals resistant to both and TSR individuals only resistant to one. Also resistance and cost (intercept on fecundity) are negatively correlated, there the slope of the correlation giving size of the cost. We can have MR as a selected trait following the approach of \citep{Coul2010} eqn. 2 and eqn. 4. We then have some proportion of all individuals that also have TSR in the next generation of seeds in the seed bank.

\subsection{overall model}

\begin{equation}
	n(g, z, t + 1) = b(g, z, t)\phi_e
\end{equation} 
Where $n(g, z, t)$ is the number of individuals of TSR genotype $g$, with MR $z$ at time $t$, $b(g, z, t)$ is the number of seeds in the seed bank of TSR genotype $g$, with MR $z$ at time $t$, $\phi_e$ is the germination probability. 

\subsection{genetic model}
We assume that TSR is controlled by a single gene, which has two versions, the susceptible version $v_0$ and the resistant version $v_r$. We assume that every individual has two copies of the TSR gene, each of which could be $v_0$ or $v_r$. This leads to three TSR genotypes, $G = \{(v_0, v_0), (v_0, v_r), (v_r, v_r)\}$, where $G$ is the set of all possible TSR gene combinations and $g_i$ is the $i^\text{th}$ copy of the gene in tuple $g$. MR on the other hand is assumed to be controlled by a large number of genes so that quantitative genetics can be used to model its evolution through time as a normal distribution of resistance values $Z(t)$ across the whole population. 
\begin{equation}\label{eq:MR_dist}
	Z(t) = N(\bar{z}(t),\sqrt{\sigma_s^2(t)})
\end{equation}
where (following \cite{Coul2010})
\begin{equation}\label{eq:MR_mean}
	\bar{z}_s(t) = \frac{\int \text{d}z~z\displaystyle\sum_{\forall g \in G}n(g, z, t)s(g, z, t)}{\int \text{d}z\displaystyle\sum_{\forall g \in G}n(g, z, t)s(g, z, t)}
\end{equation}
where $s(g, z, t)$ is the probability that an individual of TSR genotype $g$, with MR score $z$ that germinates in time $t$ will survive until flowering and seed set. $\sigma_s^2(t)$ is the variance of the distribution of MR scores, $Z(t)$, calculated as 
    
\begin{equation}\label{eq:MR_sd}
	\sigma_s^2(t) = \frac{\int \text{d}z~z^2\displaystyle\sum_{\forall g \in G}n(g, z, t)s(g, z, t)}{\int \text{d}z\displaystyle\sum_{\forall g \in G}n(g, z, t)s(g, z, t)}
\end{equation}

\subsection{seed bank and fecundity}
We assume that all seeds germinate out of the seed bank at the start of the time step. We model the number of seeds in the seed bank with MR resistance $z(t)$, TSR resistance genotype $g$ at time $t$ as the proportion of seeds already in the seed bank that don't germinate and do survive one time step (first term in Eq. \ref{eq:seedbank}) plus the number of seeds added to the seed bank at the end of the previous turn (second term in Eq. \ref{eq:seedbank}).
\begin{equation}\label{eq:seedbank}
	b(g, z, t) = b(g, z, t - 1)(1 - \phi_e)\phi_b + f(g, z, t - 1)  
\end{equation} 
Where $b(g, z, t - 1)$ is the number of seeds in the seed bank with TSR genotype $g$, herbicide resistance score $z$ in the previous time step. $b(g, z, t)$ is a normal distribution on $z$ with a mean $\bar{z}_b(t) = \bar{z}_b(t - 1) + \bar{z}_s(t)$ and variance $\sigma_b^2(t) = \sigma_b^2(t - 1) + \sigma_s^2(t)$. $(1 - \phi_e)$ is the proportion of seeds that did not germinate and $\phi_b$ is the probability that a seed survives. New seeds are produced by the fecundity function 
\begin{subequations}
\label{eq:fecund}
\begin{align}
	\displaystyle
	\label{eq:fecund_main}
	f(g, z, t) &= Z(t)\sum_{\forall g' \in G}\left(\int \text{d}z~n(g', z, t)s(g', z, t)\psi(z) \sum_{i = 1}^{i = 2} Q(g_i, g') + \mu)q(i, t)\right)\\
	\label{eq:fecund_maternal}
	Q(g_i, g') &= 
	\begin{cases}
		0.5 ~\text{if}~ g_i \in g' \\
		\mu ~\text~{otherwise}
	\end{cases}\\
	\label{eq:fecund_pollen}
	q(i, t) &= \frac{\displaystyle\sum_{\forall g^* \in G} \eta(i, g^*) \int \text{d}z~n(g^*, z, t)}{\displaystyle \sum_{\forall g^* \in G} \int \text{d}z~n(g^*, z, t)}\\
	\label{eq:pollen_freq}
	\eta(i, g^*) &= 
	\begin{cases}
	0.5~\text{if}~g_{k \neq i} \in g^* \\
	\mu ~\text{otherwise}
	\end{cases}    
\end{align} 
\end{subequations}
The seeds produced by individuals with TSR genotype $g$ and MR at time $t$ are assumed to have MR scores, $z$ that follow the normal distribution $Z(t)$ described in Eq. \ref{eq:MR_dist} (total number of seeds multiplied by $Z(t)$ in Eq. \ref{eq:fecund_main}). Because black grass is an outcrossing species individuals from other gneotypes ($g'$) can contribute gene copies to the geneotype being evaluated ($g$), thus in Eq. \ref{eq:fecund_main} the first summation term sums the seeds of TS genotype $g$ from all possible TS genotype parents ($\forall g' \in G$). The integration term in Eq. \ref{eq:fecund_main} gives the number of seeds prodced by all individuals of genotype $g'$, where $\psi(z)$ is the number of seeds produced by and individual with MR score $z$, modeled as a decresing function of $z$.
\begin{equation}\label{eq:fecund_seed_num}
	\phi(z) = \frac{\phi_\text{max}}{1 + e^{-\alpha(z - z_{50})}}
\end{equation}         
where $\phi_{max}$ is the maximum number of seeds an individual can produce, $e$ is Euler's constant, $\alpha$ is a shape parameter that affects how quickley increasing MR score, $z$ reduces feundity and $z_{50}$ is the $z$ score where fecundity is 50\% of $\phi_\text{max}$. The second summation term across gene copies $i$ gives the probability seeds with tuple $g$ will be produced by an individual of genotype $g'$. Genes copies, $g_i$, have two sources, the individual producing the seed provides half the copies of a gene, while pollen provides the other half. In Eq. \ref{eq:fecund_main} these two processes are modelled with the functions $Q(g_i, g)$, which contributes gene versions ($g_i$) from the mother, and $q(i, t)$, which contributes gene versions from pollen. $Q(g_i, g)$ is a function that returns 0.5 if the value held by copy $i$ in the target genotype $g$ ($g_i$) is in the genotype being evaluated $g'$, and returns $\mu$ otherwise. $mu$ is the probability that gene version $g'_i$ mutates gene $g_i$. This function returns 0.5 since the mother can only supply one out of two copies for each gene. The second copy of the gene is supplied from the pollen using function $q(i, t)$. We assume that the probability of getting gene version $g_i$ required to complete the tuple $g$ is the proportion of all pollen which contains that gene version. We assume that the amount of pollen produced with TSR gene version $g_i$ is proportional to the number of individuals with that gene copy. Thus, the denominator of Eq. \ref{eq:fecund_pollen} is proportional to the total amount of pollen for all genotypes (integral in denominator). The numerator of Eq. \ref{eq:fecund_pollen} is proportional to the amount of pollen that contains the gene version in the target genotype $g$, that is not supplied by the mother. It is function $\eta(i, g^*)$ that determines if this gene version will be produced by the genotype being evaluated for pollen production, $g^*$ (Eq. \ref{eq:pollen_freq}). $\eta(i, g^*)$ returns 0.5 if the version of the gene not provided by copy $g_i$ (i.e. $g_{k \neq i}$ , the copy provided by the mother) is a member of the tuple $g^*$, the genotype being evaluated for pollen production.

\subsection{survival}   
We assume that probability of survival to flowering and seed set ($s(g, z, t)$) is influenced by genotype, herbicide application and density. It is at this point that herbicide application separates the different genotypes and drives the spread of herbicide resistant genotypes.     
\begin{equation}\label{eq:survival}
	s(g, z, t) = \gamma(g, t)\phi_s(h_t, z, g) 
\end{equation}

Where $\phi_s(h_t, z, g)$ is the survival probability for an individual with genotype $g$ given the herbicide application in time step $t$, $h_t$, and $\gamma(N_x)$ is a density dependent term that reduces establishment and survival above some density threshold.  

Survival probability, $s(h_t, g)$, must be bound between 0 and 1, thus we model it as a generalised logistic function of genotype     
\begin{equation}\label{eq:herb_surv}
	s(h_t, g) =	\begin{cases}
		\frac{\beta_0 e^{-C(g)}}{1 + \beta_0 (e^{-C(g)} - 1)} &\text{:} h_t = 0 \\
		\frac{\beta_0 R(g) e^{-C(g)}}{1 + \beta_0 R(g) (e^{-C(g)} - 1)} &\text{:} h_t = 1
	\end{cases}	
\end{equation}
where $\beta_0$ is the survival probability for an individual with no resistance gene versions in their genotype, $e$ is Euler's constant, $C(g)$ is function that describes the survival costs of having genotype $g$ and $R(g)$ is a function that describe the herbicide resistance of genotype $g$.

The cost and resistance functions ($C(g)$ and $R(g)$) are how the genotype interacts with the population model through the cost of having a given genotype and the resistance that genotype provides. The survival cost of having genotype $g$ is assumed to be a linear function of the summed cost of each gene in $g$, with and intercept of 0 (implying that individuals with no expressed copies of a resistant gene have no cost)                
\begin{equation}
	\displaystyle
	\label{eq:cost_funct}	
	C(g) = \sum_{\forall j} \textbf{max}(\epsilon(\kappa_{1,g,j}), \epsilon(\kappa_{2,g,j}))\\
\end{equation}
where $\epsilon(\kappa_{k,g,j})$ is the cost, in terms of reduced survival, of having the version of gene $j$ held in copy $\kappa_{k,g,j}$ (recall that $\kappa_{k,g,j}$ can be $v_{j,0}$ or $v_{j,r}$). The resistance function, $R(g)$ is bound to be between 0 and 1, thus we use another generalised logistic function to meet these constraints.    
\begin{subequations} \label{eq:herb_attack}
\begin{align}
	\displaystyle
	\label{eq:herb_attack_logit} 
	R(g) &= \frac{\xi_0 e^{r(g)}}{1 + \xi_0 (e^{r(g)} - 1)} \\
	\label{eq:resist_score}
	r(g) &= \left(\sum_{\forall j} \textbf{max}(\rho(\kappa_{1,g,j}), \rho(\kappa_{2,g,j}))\right) + \text{N}(0, \sigma_r) \\
\end{align} 
\end{subequations} 
where $\xi_0$ is the herbicide effectiveness for a completely susceptible individual and $\rho(\kappa_{k,g,j})$ is the amount of herbicide resistance conferred by having the version of gene $j$ held in copy $\kappa_{k,g,j}$. $\text{N}(0, \sigma_r)$ is a random variable drawn from a normal distribution that gives the non-inheritable part of herbicide resistance, with $\sigma_r$ controlling the strength of non-inheritable resistance. We assume $\rho(v_0) = \epsilon(v_0) = 0$ and for versions that confer resistance we assume $\rho(v_r)$ and $\epsilon(v_r)$ are greater than 0. Thus, the $\textbf{max}({\cdot})$ function assumes that the resistance version of the gene is the dominant one for all $j$ genes. We could use other cost and resistance functions to change this behaviour. For example we could use $r(g) = \sum_{\forall j} \textbf{min}(\rho(\kappa_{1,g,j}), \rho(\kappa_{2,g,j}))$ to make resistant genes receive, or $\rho(\kappa_{1,g,j}) + \rho(\kappa_{2,g,j})$ to make the copies additive in their effect. If we want some genes to be dominant and others receive we need to replace the summation over all $j$ genes with a function that has a separate aggregator for each gene.             

Without a density dependent term any increasing population would continue growing until infinitely large. We model this density dependence with the term 
\begin{equation}\label{eq:density_dependence}
	\gamma(x, g, t) =
		\begin{cases}
			\frac{M}{\sum_{\forall g}s(h_t, g)z(x, g, t)} ~\text{if}~ \sum_{\forall g}s(h_t, g)z(x, g, t) > M \\
			1 ~\text{if}~ \sum_{\forall g}s(h_t, g)z(x, g, t) \leq M  
		\end{cases}
\end{equation}
where $\gamma(x, g, t)$ reduces establishment and survival probability at each location $x$ in response to the total number of individuals at location $x$ which are expected to establish at time $t$ (summation term in Eq. \ref{eq:density_dependence}). $s(h_t, g)$ and $z(x, g, t)$ are defined in Eq.'s \ref{eq:herb_surv} and \ref{eq:germ} respectively. $M$ is the average number of individuals that can establish at a given location. 

\bibliographystyle{/home/shauncoutts/Dropbox/shauns_paper/referencing/bes} 
\bibliography{/home/shauncoutts/Dropbox/shauns_paper/referencing/refs}

\end{document}