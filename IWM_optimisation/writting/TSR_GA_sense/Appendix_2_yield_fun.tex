\documentclass[12pt, a4paper]{article}

\usepackage{setspace, graphicx, lineno, caption, color, float}
\usepackage{amsmath}
\usepackage{amsfonts}
\usepackage{amssymb}
\usepackage{amsthm}
\usepackage{amstext} %to enter text in mathematical formulae
\usepackage[retainorgcmds]{IEEEtrantools}
\usepackage{natbib}
\usepackage{hyperref}

\usepackage[left=2.5cm,top=2.5cm,right=2.5cm, bottom=2.5cm,nohead]{geometry}
%paragraph formatting
\setlength{\parskip}{12pt}
\setlength{\parindent}{0cm}

\setcounter{equation}{0}
\renewcommand{\theequation}{S\arabic{equation}}

\begin{document}
\section*{Appendix 3:Reward Function}
The reward function measures how good a given IPM strategy is given a initial starting condition and parameter set that the model is run under. The reward function encodes the goals of a manager. We assume farmers are primarily driven by economic returns. The economic return consists of two parts, the income made from the crop and the costs of producing that crop. We assume that usual farm costs, such as buildings and machinery as constant from year to year, so we focus on gross margin, i.e. income - variable costs \citep[pp.~3]{Nix2016}.

Income from the crop in year $t$ is    
\begin{equation}\label{eq:yield}
	Y(N'', a_k^t, a_k^{t-1}) = \begin{cases} 
		W(a_k^t, a_k^{t-1})(Y_0 - Y_D N'' &~\text{if}~a_k^t = \text{wheat}\\
		W(a_k^t, a_k^{t-1})\vartheta &~\text{if}~a_k^t = \text{alt}\\
		0 &~\text{if}~a_k^t = \text{fallow}\\
	\end{cases}
\end{equation}   
where $Y_0$ is the yield of winter wheat (in \pounds $\cdot ha^{-1}$) when the density of black grass after management ($N''$) is 0, $Y_D$ is the rate at which yield decreases with increasing \textit{Alopecurus myosuroides} density. The yield function for wheat is fitted to empirical data, detailed in sub-section \textit{\nameref{Y_fun}}. We convert yield estimated from ton$\cdot ha^{-1}$ to \pounds $\cdot ha^{-1}$ using a wheat price of \pounds 146$\cdot$ton$^{-1}$, taken from \citet[pp.~9]{Nix2016}. 

The yield of a crop can be affected by planting the same crop two years in a row, largely due to species specific parasites and pathogens in the soil. We use the weighting function 
\begin{equation}
	W(a_k^t, a_k^{t-1}) = \begin{cases}
		\varpi &~\text{if}~a_k^t = a_k^{t-1}\\
		1 &~\text{otherwise}
	\end{cases}
\end{equation} 
to reduce yield if the same crop is used two times in succession, where $\varpi \in [0, 1]$ is the proportional yield achieved when the same crop is used, following we use $\varpi = 0.9$. We assume the yield of the alternative crop, $\varphi$, is not affected by black grass. The alternative crop is based on spring barley, a common spring crop used in the UK for the control of black grass. The average income from spring barley is \pounds 796/ha \citep[pp.~12]{Nix2016}.   

Costs depend on both the action chosen and non-weed control costs, such as fertilizer, seed and other sprays such as fungicides. We assume these other variable costs change with crop choice ($a_k$), but are constant from year to year within a crop choice. Thus the cost of action is $a_q$  
\begin{equation}
	C(a_q^t) = \sum_{\forall a_j \in a_q^t} c(a_j)
\end{equation}
where $c(a_j)$ is the cost of sub-action $a_j$. 

The cost for herbicide action $a_h$ is  
\begin{equation}\label{eq:herb_cost}
	c(a_h) = \begin{cases}
		0~&\text{if}~a_h = 0\\
		\eta_h~&\text{if}~a_h = 1 \bigvee a_h = 2 \\
		2\eta_h~&\text{if}~a_h = \text{both}
	\end{cases}
\end{equation}
Where $\eta_h$ is the cost of a single herbicide application to control black grass. We assume that most herbicide costs in winter wheat are associated with black grass control, and so take $\eta_h = \pounds 96$ \citep[pp.~9]{Nix2016}. 

The cost function for crop choice is 
\begin{equation}\label{eq:crop_cost}
	c(a_k) = \eta_{a_k}
\end{equation}   
where the parameter $\eta_{a_k}$ are the constant costs associated with each crop choice. The cost for winter wheat excludes herbicide costs associated with black grass control, and is taken as $\eta_\text{wheat} = $\pounds 383/ha \citep[pp.~9]{Nix2016}. We assume both a fallow rotation and the alternative crop are used, at least in part, to control black grass, and so we include all costs, including those associated with black grass control in $\eta_\text{alt} = $\pounds 273/ha \citep[pp.~12]{Nix2016} and $\eta_\text{fallow} = $ \pounds 36/ha \citep[pp.~202~and~284]{Nix2016}. The cost of the fallow rotation is based on two applications of glyphosate (a broad spectrum herbicide) to kill back grass after it has germinated.       

The cost function for plowing is 
\begin{equation}\label{eq:plow_cost}
	c(a_b) = \eta_b a_b 
\end{equation}
where $\theta_b = $ \pounds 73.96/ha is the contractor rates for deep plowing (which are assumed to combine all labour and capital costs) \citep[pp.~202]{Nix2016}, and $a_b \in \{0, 1\}$. 

Finally the cost function for spot control is assumed to increase proportionally with black grass density after other control actions have been taken so that
\begin{equation}
	c(a_s, t) = a_s\left(\eta_s^0 + \eta_s \sum_{\forall G} n'(G, t)\right)
\end{equation}  
where $\eta_s^0$ combines the costs of spot control incurred even when there is no \textit{A. myosuroides} and $\eta_s$ controls how quickly the costs of spot control increase with \textit{A. myosuroides} density. This functional form assumes the costs of spot control increase linearly with \textit{A. myosuroides} density. $a_s \in \{0, 1\}$ is a switch, so that the cost is incurred only if the action is taken.       

To explicitly link the above ground population to the reward function we define $N''(\mathbf{a}, n_0, t)$, the total above ground population after all control actions, at time $t$ given an initial population $n_0$ and a sequence of actions 
\begin{equation}
	\mathbf{a} = \{a_j^1, a_j^2, \cdots, a_j^T\}
\end{equation}	   
where $a_j^t$ is the action $a_j \in \mathbf{A}$ taken at time $t$ and $T$ is the time horizon over which management is run. We assume all returns after $T$ are ignored. The reward function is  
\begin{equation}
	R(\mathbf{a}, n_0) = \sum_{t=0}^T \gamma^t \Big( Y(N''(\mathbf{a}, n_0, t)) - C(a_j^t) \Big)
\end{equation}
where $R(\mathbf{a}, n_0)$ is the time discounted reward for action sequence $\mathbf{a}$ given starting population $n_0$, $\gamma \in [0, 1]$ is the discount rate. When $\gamma = 0$ only the reward in the first time step is considered, when $\gamma = 1$ returns in all future time steps up to $T$ are valued equally.

\subsection*{Parametrization of yield function} \label{Y_fun}
To fit the yield function we use data from 10 fields where harvesters recorded wheat yield for every 20m by 20m gird square of the field. We also have black grass density estimates for each grid square from state structured surveys \citep{Hick2018}. These density states were calibrated to plants/m$^2$ by \citet{Quee2011}: the density states were absent (0 [0--0.1667] plants/m$^2$)(median[inter-quartile range]), low (0.5000 [0.1667--1.5000] plants/m$^2$), medium (2.6666 [1.3333--4.7500] plants/m$^2$), high (5.0833 [3.0000--7.7916] plants/m$^2$), and very high (9.6666 [7.1250--13.1666] plants/m$^2$). We fit a linear yield function to the median plants/m$^2$ of each density state, $D$. 
\begin{subequations}
\label{eq:est_yield}
\begin{equation}
 Y_i \sim N(\widehat{Y_i}, \sigma_y)
\end{equation}
\begin{equation}
	\widehat{Y_i} = Y_0 + Y_0^j + (Y_D + Y_D^j)D 
\end{equation}
\end{subequations}
We assume the yield for grid square $i$ ($Y_i$, in ton$\cdot$ha$^{-1}$) is drawn from a normal distribution with standard deviation $\sigma_y$. Predicted yield ($\widehat{Y_i}$) is a linear function of black grass density, where $Y_0$ is the average winter wheat yield across all fields and $Y_0^j \sim N(0, \sigma_y^0)$ is the random effect of field $j$ on the intercept of field, drawn from a normal distribution with mean of 0 and standard deviation of $\sigma_y^0$. $Y_D$ is the change in yield when black grass density ($D$) changes by 1 plant$\cdot$m$^2$, averaged across all fields, and $Y_D^j \sim N(0, \sigma_y^D)$ is the effect of field on the relationship between black grass density and yield. This model was fit with the 'lme4' package \citep{Bate2015} in the R statistical language \citep{Rstat}. We assume our population exists in a 1ha field. Converting from plant$\cdot$m$^2$ to plants$\cdot$ha$^{-1}$ the yield loss function becomes 
\begin{equation}
	\widehat{Y} = 11.43 - 0.0000223 N'' 
\end{equation}
We use 95\% likelihood profile prediction intervals to describe uncertainty around these parameter estimates, see Appendix 2 for estimates of these limits. All costs are \pounds, to put the yield and costs on the same scale we assume a winter wheat price of \pounds 146/t \citep{Nix2016}, 
\begin{equation}
	\widehat{Y} = 1668 - 0.00326 N'' 
\end{equation}

\bibliographystyle{/Users/shauncoutts/Dropbox/shauns_paper/referencing/bes}
\bibliography{/Users/shauncoutts/Dropbox/shauns_paper/referencing/refs} 

\end{document}